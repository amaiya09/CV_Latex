%%%%%%%%%%%%%%%%%%%%%%%%%%%%%%%%%%%%%%%%%
% Medium Length Professional CV
% LaTeX Template
% Version 2.0 (8/5/13)
%
% This template has been downloaded from:
% http://www.LaTeXTemplates.com
%
% Original author:
% Trey Hunner (http://www.treyhunner.com/)
%
% Important note:
% This template requires the resume.cls file to be in the same directory as the
% .tex file. The resume.cls file provides the resume style used for structuring the
% document.
%
%%%%%%%%%%%%%%%%%%%%%%%%%%%%%%%%%%%%%%%%%

%----------------------------------------------------------------------------------------
%	PACKAGES AND OTHER DOCUMENT CONFIGURATIONS
%----------------------------------------------------------------------------------------

\documentclass{resume} % Use the custom resume.cls style
\usepackage{hyperref, textcomp, lipsum}
\usepackage{enumitem}
\hypersetup{
    colorlinks=true,
    linkcolor=blue,
    filecolor=blue,      
    urlcolor=blue,
}
\usepackage[left=0.5in,top=0.4in,right=0.5in,bottom=0.4in]{geometry} % 

\name{Amaiya Khardenavis}
\address{+1-250-878-4207 \\ \textit{\href{mailto:amaiya09@mail.ubc.ca}{amaiya09@mail.ubc.ca} \\ \href{https://www.linkedin.com/in/amaiyak}{LinkedIn} \\  \href{https://blogofthings.wixsite.com/amaiya}{Webpage}}}
\begin{document}
%----------------------------------------------------------------------------------------
%	EDUCATION SECTION
%----------------------------------------------------------------------------------------
\vspace*{-2mm}

\vspace*{-1mm}
\begin{rSection}{Education}
{\bf The University of British Columbia}, \textit{Canada} \hfill {September 2019 - Present} \\ 
\textit{Master of Applied Science (M.A.Sc.)} in Mechanical Engineering \\

\vspace*{-4mm}
{\bf University of Mumbai}, \textit{India} \hfill {August 2014 - July 2018} \\ 
\textit{Bachelor of Engineering (B.E.)} in Mechanical Engineering \\
\end{rSection}
%----------------------------------------------------------------------------------------
%	WORK EXPERIENCE SECTION
%----------------------------------------------------------------------------------------

\vspace*{-5mm}
\begin{rSection}{Research Experience}
\begin{rSubsection}{Life Cycle Management Laboratory, UBC}{September 2019 - Present}{Graduate Research Assistant under \href{https://engineering.ok.ubc.ca/about/contact/kasun-hewage/}{Kasun Hewage}}{Canada}
\item  Working on a \href{https://www.fortisbc.com/}{FortisBC} supported project for modelling the charging infrastructure for electric vehicles using mobile energy hubs for demand side management based on the \href{https://en.wikipedia.org/wiki/Vehicle-to-grid}{vehicle to grid (V2G)} concept. 
\end{rSubsection}
\vspace*{-2mm}
\begin{rSubsection}{Department of Energy Science \& Engineering, IIT Bombay}{August 2018 - July 2019}{Research Assistant under \href{http://www.ese.iitb.ac.in/~rb/}{Rangan Banerjee}}{ India}
\item  Involved in an independent review of the ambitious Mumbai-Pune \href{https://en.wikipedia.org/wiki/Hyperloop} {Hyperloop} project in collaboration with \href{http://www.pmrda.gov.in/}{Pune Metropolitan Development Authority (PMRDA)}. Performed a techno-societal assessment and determined the life cycle energy requirements for the corridor. 
\end{rSubsection}
\vspace*{-2mm}
\begin{rSubsection}{Rapid Manufacturing Laboratory, IIT Bombay}{August 2017 - April 2018}{Project Intern under \href{https://www.me.iitb.ac.in/?q=faculty/Prof.\%20K.\%20P.\%20Karunakaran}{K.P Karunakaran}}{India}
\item Conducted a literature and market survey to deem the feasibility of 3D printing for the manufacturing of drones. Implemented a computational \href{https://www.ansys.com/en-in/products/structures/topology-optimization}{topology optimization} method on \href{https://www.ansys.com/en-in/about-ansys/news-center/08-22-17-ansys-18-2-enhances-simulation-speed-accuracy}{ANSYS Workbench 18.2} to develop a \href{https://en.wikipedia.org/wiki/Micro_air_vehicle}{Micro Aerial Vehicle (MAV)} using \href{https://www.materialise.com/en/manufacturing/3d-printing-technology/fused-deposition-modeling}{fused deposition modeling} of ABS plastic. Carried out \href{https://en.wikipedia.org/wiki/Photogrammetry}{photogrammetry} simulations on \href{http://www.meshlab.net/}{MeshLab} to obtain land terrain area maps and models.

\end{rSubsection}
\vspace*{-2mm}
\begin{rSubsection}{Center for Propulsion Engineering, Cranfield University}{June 2017 - August 2017}{Visiting Researcher under \href{https://www.linkedin.com/in/dr-suresh-sampath-phd-ceng-fimeche-fie-56084118}{Suresh Sampath} and \href{https://www.cranfield.ac.uk/people/dr-theoklis-nikolaidis-727215}{Theoklis Nikolaidis}} {United Kingdom}
	
\item  Investigated the effect of changing ambient conditions on the performance curves of \href{https://www.grc.nasa.gov/www/k-12/airplane/aturbj.html}{turbojet}, \href{https://www.grc.nasa.gov/www/k-12/airplane/aturbf.html}{turbofan} and \href{https://en.wikipedia.org/wiki/Turboshaft}{turboshaft} gas turbine engines. Simulated the performance of the \href{https://en.wikipedia.org/wiki/Rolls-Royce_WR-21}{Rolls Royce WR-21} intercooled and recuperative gas turbine engine for varying operating conditions and optimised the  \href{http://blog.softinway.com/en/direct-off-design-performance-prediction-of-industrial-gas-turbine-engine/}{off-design}
parameters using \href{https://www.cranfield.ac.uk/centres/centre-for-propulsion-engineering/perseus}{Turbomatch} (\href{https://en.wikipedia.org/wiki/Fortran}{FORTRAN} based software tool developed at Cranfield University).
\end{rSubsection}
\vspace*{-2mm}
\begin{rSubsection}{Gas Turbine Research Establishment, DRDO}{June 2016 - July 2016}{Summer Intern under \href{https://www.researchgate.net/scientific-contributions/2035085680_SV_Ramanamurthy}{S.V Ramanamurthy}} {India}
\item  Designed a single stage \href{https://en.wikipedia.org/wiki/Axial_turbine}{axial turbine} at the \href{https://en.wikiversity.org/wiki/Jet_engine_design_point_performance}{design point} using the \href{http://www.circuitgrove.com/tutorials/axial-flow-compressors-mean-line-analysis}{mean line} design technique in collaboration with the Turbine Group. Generated turbine blade profiles using the \href{http://proceedings.asmedigitalcollection.asme.org/proceeding.aspx?articleid=2213902}{eleven parameter aerofoil geometry method}.
\end{rSubsection}
\vspace*{-2mm}
\begin{rSubsection}{Heat Pump Laboratory, IIT Bombay}{Dec 2015 - January 2016}{Winter Research Intern under  \href{https://www.me.iitb.ac.in/?q=faculty/Prof.\%20Milind\%20V.\%20Rane}{Milind Rane}} {India}
\item  Worked on the design of a \href{https://en.wikipedia.org/wiki/Heat_exchanger}{heat exchanger} system to maximise the heat capture for generating potable water by desalination of brackish water using waste heat from the \href{https://en.wikipedia.org/wiki/Condenser_(heat_transfer)}{condenser} of an air-conditioning system.\\
\\
\end{rSubsection}
\end{rSection}

\vspace*{-5mm}
\begin{rSection}{Publications}
\begin{itemize}[leftmargin=*]
\item\textbf{Khardenavis, A.}, Hewage, K., Perera, P., Shotorbani, A., Sadiq, R. Mobile Energy Hub Planning for Complex Urban Networks: A Robust Optimization Approach. (2020) \textit{Journal of Cleaner Production}. Under Review. 
\item Hirde, A., \textbf{Khardenavis, A.}, Banerjee, R., Bose, M., Pavan Hari, V.S.S. Sustainability Analysis of the Hyperloop Transportation System (2020) \textit{Sustainable Cities \& Societies}. Under Review. 
\item\textbf{Khardenavis, A.}, Karunakaran R. Design and Development of a Light Weight Quad-copter using Optimization Techniques. (2018) \textit{Proceedings of the Intl. Conference on Frontiers in Engineering, Applied Sciences and Technology (FEAST), National Institute of Technology, Trichy, India} 
\end{itemize}
\end{rSection}

\vspace*{-1.5mm}

\begin{rSection}{Relevant Courses}
\begin{itemize}[leftmargin=*]
\itemsep -0.5em 
\item \textbf{Graduate} - Alternative Energy Systems, Heating Ventilation \& Air Conditioning, Life Cycle Assessment \& Management, Environmental Risk Assessment Probability \& Random Processes, Multicriteria Optimisation \& Design of Experiments, Project Planning \& Control, Construction Engineering Management
\item \textbf{Undergraduate} - Mechanical Utility Systems, Thermal \& Fluid Power Engineering, Renewable Energy Sources, Finite Element Analysis, Heat Transfer, Fluid Mechanics, Thermodynamics, Internal Combustion Engines, Strength of Materials, Machine Design
\item \textbf{Online} - Getting Started with Python (Coursera), Python for Data Science \& AI (Coursera), Politics \& Economics of International Energy (Coursera), Energy \& the Earth (Coursera), Introduction to Aeronautical Engineering (edX), Space Mission Design \& Operations (edX)
\end{itemize}
\end{rSection}
\vspace*{-1mm}
\begin{rSection}{Technical Skills}
\begin{itemize}[leftmargin=*]
\itemsep -0.5em 
\item \textbf{CAD} - AutoCAD, Autodesk Inventor, SpaceClaim, SolidWorks
\item \textbf{Analysis Tools} - Ansys Workbench, Ansys Mechanical APDL, MATLAB, Minitab
\item \textbf{Coding } - Python, C, C++, HTML, Visual Basic, \LaTeX, ILOG CPLEX
\end{itemize}
\end{rSection}

\vspace*{-1mm}
\begin{rSection}{Awards \& Scholarships}
\begin{itemize}[leftmargin=*]
\itemsep -0.5em 
\item Awarded the \href{https://gradstudies.ok.ubc.ca/resources/award-opportunities/university-graduate-fellowship/}{University Graduate Fellowship}, The University of British Columbia: 2020
\item Awarded the \href{https://www.mitacs.ca/en/programs/accelerate/fellowship}{Accelerate Fellowship}, MITACS: 2019 - 2021
\item Best Undergraduate Project Poster Presentation Award: 2018
\item Best Undergraduate Thesis Award: 2018
\item Awarded the \href{http://www.inspire-dst.gov.in/scholarship.html}{Innovation   in   Science   Pursuit   for   Inspired   Research Scholarship}, Govt. of India: 2014
\end{itemize}
\end{rSection}

\vspace*{-1mm}
\begin{rSection}{Responsibilities}
\begin{rSubsection}{School of Engineering, UBC}{September 2020 - Present}{Graduate Teaching Asssistant}{Canada}
\item APSC 252: Thermodynamics
\end{rSubsection}
\begin{rSubsection}{Artium Student Residence}{May 2020 - Present}{Residence Advisor}{Canada}
\item Advising students regarding their problems and enhancing student life by organizing monthly social events. \\ \\
\end{rSubsection}
\end{rSection}
\begin{rSection}{References}
\begin{center}
\begin{tabular}{cc}
\textbf{Rangan Banerjee} & \textbf{Rehan Sadiq} \\
Professor \& Head of Department & Professor \& Associate Dean  \\
Energy Science \& Engineering, IIT Bombay & School of Engineering, UBC \\
\textit{\href{http://www.ese.iitb.ac.in/~rb/}{webpage} $\diamond$ \href{mailto:rangan@iitb.ac.in}{email}}  & \textit{\href{https://engineering.ok.ubc.ca/about/contact/rehan-sadiq/}{webpage} $\diamond$ \href{mailto:Rehan.Sadiq@ubc.ca}{email}}\\
\\
\textbf{Suresh Sampath} & \textbf{Kasun Hewage} \\
Director of Gas Turbine Systems \& Operations & Professor\\
Cranfield University & School of Engineering, UBC\\
\textit{\href{https://www.cranfield.ac.uk/centres/centre-for-propulsion-engineering/turbo-electric-systems-group}{webpage} $\diamond$ \href{mailto:s.sampath@cranfield.ac.uk}{email}}  & \textit{\href{https://engineering.ok.ubc.ca/about/contact/kasun-hewage/}{webpage} $\diamond$ \href{mailto:Kasun.Hewage@ubc.ca}{email}}\\
\\

\end{tabular}
\end{center}
\vspace*{-1mm}
\end{rSection}

%----------------------------------------------------------------------------------------

\end{document}
